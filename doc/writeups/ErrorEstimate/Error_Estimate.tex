% arara: pdflatex
% arara: bibtex
% arara: pdflatex
% arara: pdflatex
\documentclass{article}

\usepackage{graphicx}  % for including graphics
\usepackage{amsmath}   % math symbols like operator notation
\usepackage{siunitx}     % properly format quantities with units
\usepackage{fancyhdr}  % control headers
\usepackage{physics}
\usepackage{fullpage}

\title{Estimating Error for Numerical Evaluation of the Arrhenius Integral}
\author{C.D. Clark III}


\begin{document}
\maketitle

The Arrhenius
Integral model for thermal damage requires the evaluation of the integral\cite{WELCH--2011--optical-thermalresponseoflaser-irradiatedtissue}
\begin{equation}
  \label{eqn:arr}
  \Omega\qty(\tau) = \int\limits_{0}^{\tau} A e^{\frac{-Ea}{RT\qty(t)}} dt.
\end{equation}
For all but the simplest thermal profiles, this integral must be numerically evaluated. We would like to estimate the error accumulated
during numerical evaluation.

Assume that we have a thermal profile sampled at discrete times $\qty{t_i}$. If the temperature between any two
times is a monotonic function between the temperature at each time, then the accumulated damage between the
two times must be greater than the damage that would accumulate at minimum temperature over the time interval,
and the damage that would accumulate at the maximum temperature over the time interval.

Let $T_i$ be the temperature at the time $t_i$. The accumulated damage over the interval $\qty[t_i,t_{i+1}]$ will
satisfy
\begin{equation}
  A e^{\frac{-Ea}{R\min(T_i,T_{i+1})}} \tau
  \ge
  \Omega
  \ge
  A e^{\frac{-Ea}{R\max(T_i,T_{i+1})}} \tau
\end{equation}

A numerical implementation will need to check for the case that $T_0 = T_1$ and use the constant temperature quadrature. If $T_0$ and $T_1$ are stored as floating point numbers, then
we will probably need to check that they are ``close'', rather than ``equal''. Note that, for a linear temperature rise from $T_0$ and $T_1$, evaluating Equation \ref{eq:quadrature} will
to give a numerical value between $Ae^{-E_a/RT_0}(t_1 - t_0)$ and $Ae^{-E_a/RT_1}(t_1 - t_0)$. We can therefore write a limit on the maximum error in incurred by any numerical approximation to integral over duration $t_1 - t_0$.
\begin{align}
  \epsilon \le \frac{\abs{ Ae^{-E_a/RT_1}(t_1 - t_0) - Ae^{-E_a/RT_0}(t_1 - t_0)} }{Ae^{-E_a/RT_0}(t_1 - t_0)} = \abs{e^{-\frac{E_a}{R} \qty(\frac{1}{T_1} - \frac{1}{T_0})} - 1}
\end{align}
This gives a metric for deciding if $T_1$ and $T_0$ are ``close enough''. Given a tolerance, or maximum allowed error, we 
\begin{align}
  \frac{R}{E_a} \ln \qty( \epsilon + 1 ) \le \abs{ \frac{1}{T_1} - \frac{1}{T_0} }
\end{align}
For small $\epsilon$, $\ln \qty(\epsilon + 1) \approx \epsilon$,
\begin{align}
 \frac{1}{T_1} - \frac{1}{T_0} \ge \frac{R}{E_a}\epsilon
\end{align}





\bibliography{references_database}
\bibliographystyle{plain}


\end{document}
